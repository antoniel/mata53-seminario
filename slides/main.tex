\documentclass[aspectratio=169,xcolor=table]{beamer}
\usepackage{algorithm}
\usepackage{algpseudocode}
\usepackage[utf8]{inputenc}
\usepackage[T1]{fontenc}
\usepackage{lipsum, lmodern}
\usepackage{csquotes}
\usepackage{xcolor}
\usepackage[portuguese]{babel}
\usetheme{DCC}
\graphicspath{{imgs/}}

\author{Antoniel Magalhães, João Leahy, Luis Felipe}
\title{Problema do Caixeiro Viajante com Bônus e Passageiros (TSP-OBP)}
\institute{Universidade Federal da Bahia}
\date{\today}

\begin{document}

% Slide inicial: Título
\begin{frame}[plain,noframenumbering]
    \titlepage
\end{frame}

% Agenda
\begin{frame}{Agenda}
    \tableofcontents
\end{frame}

\section{Introdução}

% Introdução ao Problema
\begin{frame}{O que é o TSP-OBP e suas Variações}
    \begin{itemize}
        \item \textbf{O que é o TSP-OBP?}
        \begin{itemize}
            \item Extensão do clássico Problema do Caixeiro Viajante (TSP).
            \item Combina:
            \begin{itemize}
                \item Coleta de bônus em vértices específicos.
                \item Transporte de passageiros com restrições (capacidade, tempo, origem-destino).
            \end{itemize}
            \item Aplicações: 
            \begin{itemize}
                \item Logística e transporte sob demanda.
                \item Roteirização de veículos com incentivos.
            \end{itemize}
        \end{itemize}
        \item \textbf{Variações do Problema:}
        \begin{itemize}
            \item \textbf{PCV-CB (Coleta de Bônus)}:  
            Visitam-se locais para coletar bônus, equilibrando custos e prêmios.
            \item \textbf{PCV-P (Passageiros)}:  
            Integra transporte de passageiros, compartilhando custos da viagem.
            \item \textbf{Problema do Orienteering (OP)}:  
            Maximização de prêmios coletados dentro de um limite de tempo.
        \end{itemize}
    \end{itemize}
\end{frame}

\section{Fundamentação Teórica}


% Complexidade NP-difícil
\begin{frame}{Complexidade NP-difícil}
    \begin{itemize}
        \item O TSP-OBP é um problema \textbf{NP-difícil}, o que implica que:
        \begin{itemize}
            \item Não existe algoritmo conhecido que garanta a solução ótima em tempo polinomial para todas as instâncias.
            \item A inclusão de bônus e passageiros adiciona novas dimensões de complexidade combinatória.
        \end{itemize}
        \item Consequências da complexidade:
        \begin{itemize}
            \item Soluções exatas são inviáveis para instâncias grandes.
            \item É necessário o uso de \textbf{heurísticas e meta-heurísticas}, que buscam soluções boas em tempo razoável.
        \end{itemize}
    \end{itemize}
\end{frame}

% Função Objetivo
\begin{frame}{Função Objetivo no TSP-OBP}
    \begin{itemize}
        \item A função objetivo busca equilibrar dois objetivos principais:
        \begin{itemize}
            \item \textbf{Maximizar bônus coletados:} Incentiva visitar vértices com prêmios adicionais.
            \item \textbf{Minimizar custos de deslocamento:} Reduz distâncias percorridas e custos operacionais.
        \end{itemize}
        \item Formulação matemática:
        \[
        \text{max} \, \sum_{i \in V} b_i y_i - \alpha \sum_{(i,j) \in E} c_{ij} x_{ij}
        \]
        \item Importância em otimização multiobjetivo:
        \begin{itemize}
            \item Permite modelar conflitos entre diferentes metas, como eficiência e rentabilidade.
            \item Essencial para resolver problemas com múltiplas prioridades, como logística e transporte.
        \end{itemize}
    \end{itemize}
\end{frame}

\section{Metodologia}
% Métodos Considerados
\begin{frame}{Métodos Considerados e Não Selecionados}
    \begin{itemize}
        \item \textbf{Heurística de Carregamento (HC)}:
        \begin{itemize}
            \item Verifica a disponibilidade de assentos para passageiros em um determinado trecho.
            \item Usada para determinar se um passageiro pode ser embarcado em um trecho da rota.
        \end{itemize}
        \item \textbf{NVH (Neighborhood Variable Heuristic)}:
        \begin{itemize}
            \item Busca a melhor solução explorando diferentes vizinhanças de uma solução inicial.
        \end{itemize}
        \item \textbf{VND (Variable Neighborhood Descent)}:
        \begin{itemize}
            \item Utiliza diferentes estruturas de vizinhança para explorar o espaço de soluções.
        \end{itemize}
    \end{itemize}
\end{frame}

\section{Algoritmos Selecionados}

% Algoritmo GRASP
\begin{frame}{Abordagem Baseada em GRASP}
    \begin{itemize}
        \item \textbf{Fase de Construção Semi-Gulosa}:
        \begin{itemize}
            \item Seleção de vértices com base em custos e bônus.
            \item Adição iterativa à rota.
        \end{itemize}
        \item \textbf{Fase de Busca Local com VND}:
        \begin{itemize}
            \item Troca, remoção e reinserção de passageiros.
            \item Refinamento da solução inicial.
        \end{itemize}
    \end{itemize}
\end{frame}

\section{Pseudocódigo}

% Pseudocódigo GRASP
\begin{frame}{Pseudocódigo do GRASP para o TSP-OBP}
    \begin{algorithm}[H]
        \caption{GRASP para o TSP-OBP}
        \begin{algorithmic}
            \Require Parâmetros: $max\_iter$, $max\_no\_improve$
            \Ensure Melhor solução encontrada
            \State $best\_solution \gets \emptyset$
            \State $best\_cost \gets \infty$
            \For{$i \gets 1$ até $max\_iter$}
                \State $solution \gets ConstructSolution()$
                \State $solution \gets LocalSearch(solution)$
                \If{$Cost(solution) < best\_cost$}
                    \State $best\_solution \gets solution$
                    \State $best\_cost \gets Cost(solution)$
                \EndIf
            \EndFor
            \State \Return $best\_solution$
        \end{algorithmic}
    \end{algorithm}
\end{frame}

\section{Outras Perspectivas Algorítmicas}

% Heurísticas e Meta-Heurísticas
\begin{frame}{Heurísticas e Meta-Heurísticas Adicionais}
    \begin{itemize}
        \item \textbf{Heurística Gulosa}:
        \begin{itemize}
            \item Simplicidade, mas frequentemente limitada a soluções subótimas.
        \end{itemize}
        \item \textbf{Simulated Annealing (SA)}:
        \begin{itemize}
            \item Baseado no recozimento metálico.
            \item Equilibra exploração e intensificação com alto custo computacional.
        \end{itemize}
        \item \textbf{Algoritmos Genéticos (GA)}:
        \begin{itemize}
            \item Usa evolução natural para buscar soluções.
            \item Requer ajuste fino de parâmetros.
        \end{itemize}
        \item \textbf{Ant Colony Optimization (ACO)}:
        \begin{itemize}
            \item Baseado no comportamento de formigas reais.
            \item Eficiente em problemas combinatórios, como roteamento e escalonamento.
            \item Requer ajustes de parâmetros para equilíbrio entre exploração e intensificação.
        \end{itemize}
    \end{itemize}
\end{frame}

\section{Conclusão}

% Conclusão e Próximos Passos
\begin{frame}{Conclusão}
    \begin{itemize}
        \item O TSP-OBP é uma extensão relevante do TSP clássico, combinando desafios adicionais como a coleta de bônus e transporte de passageiros.
        \item A complexidade do problema exige o uso de algoritmos híbridos e meta-heurísticas, como GRASP e VND, que demonstram eficiência prática.
        \item Aplicações em logística, transporte sob demanda e roteirização tornam o problema altamente relevante em contextos reais.
        \item A abordagem discutida ilustra como é possível equilibrar múltiplos objetivos, como custos, prêmios e restrições de transporte.
    \end{itemize}
\end{frame}

\section{Referências}

% Referências
\begin{frame}{Referências}
    \begin{itemize}
        \item Lopes Filho, J. G. (2019). Problema do Caixeiro Viajante com Coleta Opcional de Bônus, Tempo de Coleta e Passageiros. Tese de Doutorado, Universidade Federal do Rio Grande do Norte, Natal-RN.
        
        \item Carvalho, M. R. (2022). Métodos Heurísticos para o TSP-OBP. Journal of Combinatorial Optimization.
        
        \item Carnielli, W. and Epstein, R. (2017). Computabilidade e Funções Computáveis. UNESP.
        
        \item Goldbarg, M. and Goldbarg, E. (2012). Grafos: Conceitos, algoritmos e aplicações. Elsevier.
    \end{itemize}
\end{frame}

\end{document}