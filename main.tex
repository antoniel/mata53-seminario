%  #region preâmbulo
\documentclass[12pt, a4paper]{report}
\usepackage[top=3cm,left=3cm,right=2cm,bottom=2cm]{geometry}
\linespread{1.3}
\setlength{\parindent}{1.25cm}
\usepackage{indentfirst}
\usepackage[utf8]{inputenc}
\usepackage[brazil]{babel}
\usepackage{amsmath}
\usepackage{amsthm}
\usepackage{amsfonts}
\usepackage{amssymb}
\usepackage{graphicx}
\usepackage{color}
\usepackage{multicol}
\usepackage[normalem]{ulem}
\usepackage{wrapfig}
\usepackage{caption}
\usepackage{fancybox}
\usepackage[pdfstartview=FitH]{hyperref}
\usepackage{subfigure}
\bibliographystyle{plain}

\graphicspath{{Figuras/}}

\renewcommand{\theenumii}{\alph{enumii}}
\DeclareMathOperator{\sen}{sen}
\DeclareMathOperator{\tg}{tg}
\DeclareMathOperator{\arctg}{arctg}
\DeclareMathOperator{\cotg}{cotg}
\DeclareMathOperator{\agm}{agm}

\newtheorem{thm}{Teorema}[section]
\newtheorem{dfn}{Definição}[section]
\newtheorem{prob}{Problema}[section]
\newtheorem{cor}{Corolário}[section]
\newtheorem{prop}{Proposição}[section]
\newtheorem{lem}{Lema} [section]

\newcounter{contar}
%  #endregion preâmbulo

% Variáveis 
\newcommand{\nomeUniversidade}{Universidade Federal da Bahia}
\newcommand{\nomeInstituto}{Instituto de Computação}
\newcommand{\nomeCurso}{MATA53 - Teoria dos grafos}
\newcommand{\nomeProfessor}{Islame Felipe da Costa Fernandes}
\newcommand{\nomeGrupo}{\sc{\large{Antoniel Magalhães}} \\
\sc{\large{João Leahy}} \\
\sc{\large{Luis Felipe}}}
\newcommand{\titulo}{\sc{\Large{Relatório: Problema do caixeiro viajante com bônus e passageiros}}}

\begin{document}

% #region capa
\pagestyle{empty}
\begin{center}
\includegraphics[height=2.5cm]{UFBA.jpg}
\hspace{2cm}
\end{center}

\begin{center}
\sc{\large{\nomeUniversidade}} \\
\sc{\large{\nomeInstituto}} \\
\sc{\small{\nomeCurso}} \\

\vspace{4cm}

\titulo

\vspace{4.5cm}

\nomeGrupo


\vspace{5.5cm}

\textbf{Salvador - Bahia} \\
\today
\end{center}
% #endregion capa

% #region folha de rosto
\newpage
\begin{center}
\titulo

\vspace{4cm}

\nomeGrupo
\end{center}

\vspace{4cm}

\begin{flushright}
\begin{minipage}{8.6cm}
Estudo dirigido entregue ao professor \nomeProfessor
como método avaliativo da disciplina \nomeCurso


\end{minipage}
\end{flushright}
 
\vspace{8cm}


\begin{center}
\textbf{Salvador - Bahia} \\
\today
\end{center}

% #endregion folha de rosto

% #region Índice
\newpage
\tableofcontents
\thispagestyle{empty}
% #endregion Índice



% #endregion Introdução
\newpage
\setcounter{page}{1}
\section*{Introdução}
O Problema do Caixeiro Viajante com Coleta de Bônus e Passageiros (TSP-OBP) é uma variação do clássico Problema do Caixeiro Viajante (TSP). Ele se distingue pela incorporação de coleta de bônus opcionais e pelo transporte de passageiros, adicionando complexidade ao problema original. O TSP clássico busca a rota mais curta para visitar um conjunto de cidades e retornar ao ponto de partida, enquanto o TSP-OBP estende essa ideia, permitindo a coleta de bônus em locais específicos e o transporte de passageiros entre pontos predeterminados.
\subsection*{Componentes do TSP-OBP}
\begin{itemize} \item \textbf{Coleta de bônus:} O caixeiro pode escolher visitar nós opcionais para coletar bônus, o que pode aumentar o lucro total, apesar de potencialmente aumentar a distância percorrida. Existe um equilíbrio entre o custo da viagem e o lucro obtido pelos bônus. \item \textbf{Transporte de passageiros:} O problema inclui a restrição de pegar e deixar passageiros em locais predefinidos, adicionando uma camada extra de complexidade ao planejamento da rota. Os passageiros devem ser transportados dentro de janelas de tempo específicas, garantindo que sua inclusão na rota não cause grandes interrupções no itinerário. \end{itemize}
\subsection*{Aplicações e Benefícios}
O TSP-OBP possui aplicações relevantes na otimização de serviços de entrega de última milha, com benefícios como redução de custos e aumento da satisfação do cliente. Em 2025, a entrega de última milha é responsável por até 41% dos custos totais de logística, tornando a otimização de rotas crucial. A integração do TSP-OBP permite reduzir o consumo de combustível, o tempo de deslocamento e os custos gerais de entrega. Espera-se que o mercado de soluções de entrega de última milha cresça significativamente, de $725.01 milhões em 2024 para $2092.95 milhões em 2031. Ao incorporar o TSP-OBP, empresas podem otimizar suas rotas de entrega, garantindo eficiência e pontualidade. O TSP-OBP também aborda desafios como congestionamento de tráfego e flutuações na demanda de entrega.
\subsection*{Integração com Dados em Tempo Real}
A integração de dados em tempo real, como atualizações de tráfego e solicitações inesperadas de passageiros, é fundamental para a otimização de rotas. A otimização dinâmica de rotas adapta-se a mudanças súbitas no tráfego, condições climáticas e pedidos de última hora. Utilizando algoritmos avançados e dados em tempo real, as empresas podem reduzir custos, minimizar o impacto ambiental e aumentar a satisfação do cliente.
\subsection*{Abordagens de Solução}
\subsubsection*{Programação Linear Inteira Mista (MILP)}
O TSP pode ser formulado usando Programação Linear Inteira Mista (MILP) para incorporar restrições adicionais, como coleta de bônus e passageiros. A formulação MILP envolve a definição de restrições de correspondência para assegurar que cada vértice tenha grau dois no ciclo, juntamente com restrições de eliminação de subtour (SECs) para manter a conectividade. No entanto, devido ao número exponencial de SECs, formulações alternativas, como a abordagem de Miller-Tucker-Zemlin (MTZ), são utilizadas. As vantagens do MILP incluem flexibilidade e capacidade de produzir soluções ótimas ou quase ótimas para instâncias de tamanho moderado. As desvantagens incluem complexidade computacional e crescimento exponencial do problema com o aumento do número de cidades.
\subsubsection*{Métodos Heurísticos}
Métodos heurísticos são cruciais para resolver o TSP com coleta de bônus e passageiros. O Método de Agrupamento (G-M) é usado para lidar com o Problema de Coleta e Entrega de Passageiros (PPDP), minimizando custos de viagem e satisfazendo restrições de passageiros. Sistemas de Colônia de Formigas (ACS) otimizam o Problema do Caixeiro Viajante Dinâmico (DTSP), incorporando novas paradas sem reiniciar todo o processo de otimização. A otimização combinatória neural utiliza aprendizado por reforço para resolver problemas como o TSP.
\subsubsection*{Algoritmos Genéticos (GAs)}
Algoritmos genéticos (GAs) são técnicas de otimização inspiradas na seleção natural. Eles são eficazes para resolver o TSP, explorando um vasto espaço de soluções e evitando ótimos locais. GAs evoluem uma população de soluções candidatas, selecionando os indivíduos mais aptos para reprodução usando operações de cruzamento e mutação. Em comparação com MILP, os GAs podem lidar com problemas maiores e encontrar soluções aproximadas rapidamente.
\subsection*{Implementações no Mundo Real}
O TSP tem aplicações em logística, transporte e manufatura. Implementações incluem otimização de rotas de entrega usando algoritmos avançados e tecnologias como computação em nuvem e processamento paralelo. Projetos utilizam bibliotecas Python como SciPy, NetworkX e Matplotlib para calcular a rota mais curta entre locais de entrega. Outros projetos consideram restrições como janelas de tempo de entrega, capacidades de caminhões e requisitos especiais de pacotes.
% #endregion Introdução

% Capítulo 2: Fundamentação Teórica
\chapter{Fundamentação Teórica}

\section{Problema do Caixeiro Viajante (TSP)}
O Problema do Caixeiro Viajante consiste em encontrar a menor rota que visita um conjunto de cidades exatamente uma vez e retorna ao ponto de origem. É um problema clássico de otimização combinatória, amplamente estudado na teoria da computação.

\section{Extensões do TSP}
O TSP-OBP é uma extensão do TSP, incorporando os seguintes elementos:
\begin{itemize}
    \item \textbf{Coleta de Bônus}: Cada local visitado oferece um prêmio opcional. O objetivo é maximizar o bônus coletado, respeitando restrições de tempo ou custo.
    \item \textbf{Passageiros}: O transporte de passageiros pode ser incluído na rota, com cada passageiro contribuindo com uma recompensa adicional.
\end{itemize}

\section{Definição Formal do Problema}
Dado um conjunto de nós (cidades) $N$, uma matriz de custos $C = [c_{ij}]$, e bônus $b_i$ associados a cada nó $i$, o objetivo é determinar uma rota $R$ que:
\begin{itemize}
    \item Minimiza o custo total de viagem $C(R)$;
    \item Maximiza a soma dos bônus coletados $B(R)$;
    \item Satisfaz restrições, como limite de tempo e capacidade para passageiros.
\end{itemize}

% Capítulo 3: Modelagem Matemática
\chapter{Modelagem Matemática}

\section{Variáveis de Decisão}
\begin{itemize}
    \item $x_{ij} \in \{0, 1\}$: Indica se a aresta $(i, j)$ é percorrida.
    \item $y_i \in \{0, 1\}$: Indica se o bônus do nó $i$ foi coletado.
    \item $p_i \in \{0, 1\}$: Indica se um passageiro foi embarcado no nó $i$.
\end{itemize}

\section{Função Objetivo}
\[
\text{Maximizar: } \sum_{i \in N} b_i y_i - \sum_{(i,j) \in E} c_{ij} x_{ij}
\]
\section{Restrições}
\begin{enumerate}
    \item Cada nó deve ser visitado exatamente uma vez:
    \[
    \sum_{j \in N} x_{ij} = 1, \quad \forall i \in N
    \]
    \[
    \sum_{i \in N} x_{ij} = 1, \quad \forall j \in N
    \]
    \item Limites de tempo:
    \[
    \sum_{(i,j) \in E} c_{ij} x_{ij} \leq T_{\text{máximo}}
    \]
    \item Capacidade de transporte:
    \[
    \sum_{i \in N} p_i \leq \text{Capacidade}
    \]
\end{enumerate}

% Capítulo 4: Métodos de Resolução
\chapter{Métodos de Resolução}

\section{Métodos Exatos}
\begin{itemize}
    \item \textbf{Branch and Bound}: Explora todas as soluções possíveis de forma sistemática.
    \item \textbf{Programação Linear Inteira (ILP)}: Resolve a formulação matemática do problema usando técnicas de otimização.
\end{itemize}

\section{Heurísticas}
\begin{itemize}
    \item \textbf{Algoritmos Gulosos}: Constroem uma solução passo a passo, escolhendo a melhor opção local em cada etapa.
    \item \textbf{GRASP (Greedy Randomized Adaptive Search Procedure)}: Combina buscas gulosas com elementos aleatórios.
\end{itemize}

\section{Metaheurísticas}
\begin{itemize}
    \item \textbf{Algoritmos Genéticos}: Inspirados na evolução natural.
    \item \textbf{Simulated Annealing}: Baseado no recozimento térmico.
    \item \textbf{Ant Colony Optimization}: Inspirado no comportamento de colônias de formigas.
\end{itemize}

% Capítulo 5: Resultados e Discussão
\chapter{Resultados e Discussão}

\section{Resultados Computacionais}
Apresente os resultados obtidos ao resolver instâncias do problema, comparando diferentes métodos em termos de:
\begin{itemize}
    \item Tempo de execução;
    \item Qualidade da solução;
    \item Eficiência computacional.
\end{itemize}

\section{Discussão}
Discuta os trade-offs entre os métodos utilizados e os impactos práticos dos resultados.

% Capítulo 6: Conclusão
\chapter{Conclusão}
Resumo dos principais resultados e contribuições do trabalho. Apresente sugestões para trabalhos futuros, como:
\begin{itemize}
    \item Estudo de novas heurísticas para grandes instâncias.
    \item Extensão do modelo para múltiplos veículos.
    \item Aplicações em cenários reais, como planejamento logístico.
\end{itemize}

Neste trabalho visitamos teorias que fundamentam a base da computação, passando primeiro pelo concceito de funções recursivas sua importância e definições. posteriormente discorremos sobre o Lambda-cálculo e sua contextualização, e por fim relacionamos tudo apresentado com Máquinas de Turing que foi o objeto de estudo principal da disciplina.\\
É importante perceber que embora as Máquinas de Turing seja o modelo de computação preferido para o estudo em computabilidade, existem modelos clássicos dos quais o estudo continua relevante e necessário para o avanço da computação como ciência.

%-------------Bibliografia------------------
\newpage
\renewcommand{\refname}{Referências Bibliográficas}
\addcontentsline{toc}{chapter}{Referências Bibliográficas}
\bibliography{Bibliografia}
\nocite{gomes2016, gomes2018, salvador, machado, ramos, carnielli}


\end{document}
